\section{Introducción}
Los sistemas cuánticos deben entenderse como sistemas abiertos debido a su inevitable interacción con el entorno, que influye significativamente en su dinámica. Dado que el aislamiento perfecto es imposible y la descripción microscópica completa del entorno resulta impráctica o irrelevante, la teoría de sistemas cuánticos abiertos se vuelve esencial. Este enfoque simplifica el análisis mediante herramientas probabilísticas, permitiendo tratar sistemas complejos con múltiples grados de libertad y enfocándose en información relevante para describir su evolución.

Cuando se tiene un sistema cuántico cerrado, según la ecuación de Schrödinger los autoestados del Hamimtoniano son estados estacionarios y la evolución temporal es unitaria, por lo tanto solo agregan una fase global a los estados pero no produce cambios a otros estados. A su vez, cuando se quieren describir situaciones más complejas considerando estados mixtos, se utiliza la matriz de densidad \cite{blum2012density}. Esta matriz es una generalización de los estados puros, y se puede escribir como 

\begin{equation}    \label{eq: dm}
    \rho = \sum_i p_i |\psi_i\rangle \langle \psi_i|
\end{equation} 

\noindent donde \(p_i\) es la probabilidad de que el sistema esté en el estado \(|\psi_i\rangle\). Esta probabilidad no es cuántica, sino que es una probabilidad clásica que surge de nuestro desconocimiento del estado cuántico. Notar que esta probabilidad es diferente a la probabilidad cuántica, proveniente de una superposición de estados.

Los elementos de la diagonal de la \(\rho\) se conocen como \textbf{poblaciones} y los elementos fuera de la diagonal como \textbf{coherencias}. Notar que las coherencias indican superposición entre estados cuánticos. Además, se puede demostrar que la matriz de densidad (Ecuación \ref{eq: dm}) es hermítica y tiene traza unitaria \cite{blum2012density}.

% Por definición, se tienen estados puros $\iff Tr(\rho^2)=1$.

La evolución temporal de la matriz de densidad sigue la ecuación de Liouville-von Neumann dada por \cite{blum2012density,BRE02}

\begin{equation} \label{eq: Liouville-von Neumann}
    \frac{d\rho}{dt} = \frac{1}{i\hbar}[\mathcal{H},\rho]
\end{equation}

\noindent donde \(\mathcal{H}\) es el Hamiltoniano del sistema. Notar que esta evolución temporal es unitaria, por lo tanto los autoestados del Hamiltoniano son estados estacionarios.

Si consideramos una situación más realista, en donde el sistema (S) se encuentra embebido en un entorno (E), entonces la matriz de densidad del universo (U) se puede escribir como

\begin{equation} \label{eq: dm_u}
    \rho_u = \rho_s \otimes \rho_e,
\end{equation}

\noindent donde \(\rho_s \in \mathcal{H}_s\) y \(\rho_e \in \mathcal{H}_e\) son las matrices de densidad del sistema y del entorno respectivamente en sus respectivos espacios de Hilbert, por lo tanto, \(\rho_u \in \mathcal{H}_s \otimes \mathcal{H}_e = \mathcal{H}_u\).

La evolución temporal de la matriz de densidad del universo es unitaria y se puede obtener mediante la Ecuación \ref{eq: Liouville-von Neumann} como

\begin{equation} \nonumber
    \partial_t \rho_u =  \frac{1}{i\hbar} [\mathcal{H}, \rho_u].
\end{equation}

Como se quiere obtener la evolución temporal de la matriz de densidad del sistema, se tiene que realizar la traza parcial sobre el entorno, es decir \(\rho_s (t) = \text{tr}_e (\rho_u (t))\).  Si se escribe el Hamiltoniano del universo como el producto tensorial de los Hamiltonianos del sistema y del entorno más un término de interacción, se puede demostrar que la evolución temporal de la matriz de densidad del sistema es \cite{BRE02}

\begin{align} \label{eq: Lindblad}
    \rho_s (t) &= \frac{1}{i\hbar} \left[ \mathcal{H}_s, \rho_s \right] \nonumber \\
    &\quad + \sum_n \left(C_n \rho C_n^\dagger - \frac{1}{2}\left\{C_nC_n^\dagger, \rho \right\} \right)
\end{align}

\noindent donde \(C_n\) son llamados operadores de colapso o de Lindbland. La ecuación \ref{eq: Lindblad} es la ecuación de Lindblad, que describe la evolución temporal de la matriz de densidad del sistema en presencia de un entorno.

Los operadors de colapso son operadores que describen la interacción del sistema con el entorno. Si bien el entorno puede ser muy complejo o contener infinitos grados de libertad, puede suponerse una iteracción efectiva dada por un conjunto finito de operadores. Usualmente, los operadores de colapso se escriben como
\(C_n = \sqrt{\gamma_n} O_n\), donde \(\gamma_n\) es una tasa efectiva de colapso y \(O_n\) es un operador del sistema.

En este trabajo se resuelve numéricamente la ecuación de Lindblad para sistemas de dos niveles, conocidos como qubits, de gran relevancia en la computación cuántica. La sección \ref{sec: metodologia} detalla el algoritmo empleado para la resolución numérica. En la sección \ref{sec: one_qubit}, se analizan los efectos de relajación, excitación y coherencia en un qubit. Posteriormente, en la sección \ref{sec: two_qubits}, se estudia la evolución temporal de dos qubits entrelazados, incluyendo la evolución de la concurrencia definida por Wootters \cite{Wootters} y el fenómeno de ``muerte súbita del entrelazamiento'' (Entanglement Sudden Death, ESD) \cite{ESD}. Finalmente, en la sección \ref{sec: conclusion}, se resumen los resultados obtenidos y su relevancia en el contexto del trabajo. 

