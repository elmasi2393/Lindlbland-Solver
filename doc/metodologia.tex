\section{Implementación} \label{sec: metodologia}

La implementación numérica de la ecuación de Lindblad se desarrolló en \texttt{C++} y está disponible en el \href{}{enlace} correspondiente. Para el manejo eficiente de matrices y vectores complejos, se utilizó la librería Eigen \cite{eigenweb}. El diseño del código adoptó una sintaxis familiar para físicos, inspirada en la notación de la librería QuTiP \cite{qutip1,qutip2} de \texttt{Python}.

Para facilitar el manejo de las operaciones matemáticas, se definieron alias como \texttt{QuantumMatrix} y \texttt{QuantumValue}, destinados a representar matrices, vectores y valores escalares complejos de forma intuitiva. El programa se estructuró en torno a la clase abstracta \texttt{QobjEvo}, que sirve como interfaz genérica para implementar el Hamiltoniano y los operadores de colapso. Esto permite definir distintos sistemas físicos mediante la herencia de esta clase, garantizando un correcto funcionamiento en la resolución numérica de la ecuación de Lindblad.

Para resolver la ecuación de Lindblad \ref{eq: Lindblad}, se implementó una función principal, \texttt{mesolve}, que utiliza el método de Runge-Kutta de cuarto orden para obtener la evolución temporal de la matriz de densidad del sistema. Esta función toma como entrada el Hamiltoniano, un vector que contiene los límites temporales, el intervalo de tiempo, el estado inicial del sistema como una matriz de densidad y un vector de operadores de colapso. Su salida es un vector de matrices de densidad que describen la evolución del sistema a lo largo del tiempo.

Además, el código incluye la función \texttt{save\_states}, que permite exportar las matrices de densidad calculadas junto con información como los límites temporales, el paso de integración y las dimensiones del sistema. Los datos se guardan en un formato estructurado que facilita su análisis y visualización con herramientas como \texttt{Python} u otros lenguajes de programación.

% En conjunto, esta implementación ofrece una herramienta versátil para estudiar sistemas abiertos en mecánica cuántica, proporcionando una base sólida para explorar fenómenos como la disipación, la decoherencia y la dinámica no unitaria de sistemas cuánticos. Su diseño modular también permite futuras extensiones, como la incorporación de nuevos algoritmos de integración o modelos físicos adicionales.
