\section{Consluciones} \label{sec: conclusion}

Se implementó un método para resolver la ecuación de Lindbland de manera numérica, utilizando el método de Runge-Kutta de cuarto orden. Se desarrolló un código en \texttt{C++} que permite estudiar la evolución temporal de diversos sistemas cuánticos definidos por el usuario de manera flexible y siguiendo la notación de la librería QuTiP de \texttt{Python}. 

% analisis de 1 qubit -> analizar procesos de relajación, excitación y decoherencia y contrastar con resultados analíticos
Se estudiaron los procesos de relajación, excitación y desfasaje en sistemas de un qubit y se comparó con resultados obtenidos de manera analítica. Se obtuvo que el método es capaz de reproducir la evolución temporal de la matriz de densidad de manera precisa, validando su eficacia para describir la dinámica de sistemas cuánticos simples. Sin embargo, no se analizaron sistemas en donde se tengan soluciones que varíen de manera abrupta, lo que implicaría un cambio dinámico en el paso temporal o la implementación de otro método numérico.

% Analisis de 2 qubit -> dechoerencia en qubits idénticos independiente sasociados a baños identicos no produce ESD en estados de Bell. Se estudió la dinámica y comparó con resultados analíiticos y se obtuvo un tiempo característico
Se estudió la decoherencia en sistemas de dos qubits idénticos acoplados a baños térmicos independientes y se analizó el proceso de muerte súbita del entrelazamiento (ESD) mediante la métrica de la concurrencia definida por Woottters. Se obtuvo que, bajo estas consideraciones, los estados de Bell no alcanzan concurrencia 0 a tiempo finito, por lo que no producen ESD. Sin embargo, se pudieron definir condiciones en las cuales se produce este fenómeno, además de un tiempo característico. Estos resultados se pudieron contrastar con los resultados numéricos obtenidos con el método desarrollado, lo que verifica su validez para estudiar sistemas cuánticos más complejos.